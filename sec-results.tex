\newcommand{\textretrieval}{\preprintonly{textretrieval-001}%
    \finalonly{textretrieval-001}%
    \submissiononly{moocname1-00X}}

\newcommand{\sustain}{\preprintonly{sustain-001}%
    \finalonly{sustain-001}%
    \submissiononly{moocname2-00X}}

\newcommand{\UIUC}{\preprintonly{UIUC}%
    \finalonly{UIUC}%
    \submissiononly{(redacted University)}}

\section{Results}
To qualitatively evaluate our model, we can look at the latent state
representations we learn by fitting the model to some MOOC log data. We can
also inspect the transition matrix between the latent states discovered by
the model.

Specifically, we look at the MOOC logs associated with two different
Coursera MOOCs offered by \UIUC{}: \textretrieval{} and \sustain{}.  The
\textretrieval{} MOOC represents a highly technical computer science course,
where the \sustain{} MOOC is more representative of a humanities course. We
picked these two MOOCs because of their vastly different content domains.

\subsection{Latent State Representations}
% highlight interesting latent states discovered; make arguments that
% motivate why it's useful to have a Markov model as the latent state
% representation (capturing transitions between actions is useful and
% more meaningful than just using a discrete distribution as the latent
% state representation); show how the learned state representations differ
% between the two MOOCs

\subsection{Varying the Number of Latent States}
% show what happens as we vary k (particularly interesting to see 2 -> 3
% hidden state change, as the forum "topic" appears suddenly); argue that
% this allows the model to capture behavior patterns at different levels of
% granularity

\subsection{Transitions Between Latent States}
% using k=6, show that we can uncover interesting patterns in how students
% transition between the different latent states; argue that this discovery
% is interesting and unique to our formulation and could not be discovered
% before
