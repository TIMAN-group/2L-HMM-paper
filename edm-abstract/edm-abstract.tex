% THIS IS SIGPROC-SP.TEX - VERSION 3.1
% WORKS WITH V3.2SP OF ACM_PROC_ARTICLE-SP.CLS
% APRIL 2009
%
% It is an example file showing how to use the 'acm_proc_article-sp.cls' V3.2SP
% LaTeX2e document class file for Conference Proceedings submissions.
% ----------------------------------------------------------------------------------------------------------------
% This .tex file (and associated .cls V3.2SP) *DOES NOT* produce:
%       1) The Permission Statement
%       2) The Conference (location) Info information
%       3) The Copyright Line with ACM data
%       4) Page numbering
% ---------------------------------------------------------------------------------------------------------------
% It is an example which *does* use the .bib file (from which the .bbl file
% is produced).
% REMEMBER HOWEVER: After having produced the .bbl file,
% and prior to final submission,
% you need to 'insert'  your .bbl file into your source .tex file so as to provide
% ONE 'self-contained' source file.
%
% Questions regarding SIGS should be sent to
% Adrienne Griscti ---> griscti@acm.org
%
% Questions/suggestions regarding the guidelines, .tex and .cls files, etc. to
% Gerald Murray ---> murray@hq.acm.org
%
% For tracking purposes - this is V3.1SP - APRIL 2009

\documentclass{edm_template}

\begin{document}

\title{Modeling MOOC Student Behavior With Two-Layer Hidden Markov Models}

% You need the command \numberofauthors to handle the 'placement
% and alignment' of the authors beneath the title.
%
% For aesthetic reasons, we recommend 'three authors at a time'
% i.e. three 'name/affiliation blocks' be placed beneath the title.
%
% NOTE: You are NOT restricted in how many 'rows' of
% "name/affiliations" may appear. We just ask that you restrict
% the number of 'columns' to three.
%
% Because of the available 'opening page real-estate'
% we ask you to refrain from putting more than six authors
% (two rows with three columns) beneath the article title.
% More than six makes the first-page appear very cluttered indeed.
%
% Use the \alignauthor commands to handle the names
% and affiliations for an 'aesthetic maximum' of six authors.
% Add names, affiliations, addresses for
% the seventh etc. author(s) as the argument for the
% \additionalauthors command.
% These 'additional authors' will be output/set for you
% without further effort on your part as the last section in
% the body of your article BEFORE References or any Appendices.

\numberofauthors{2} %  in this sample file, there are a *total*
% of EIGHT authors. SIX appear on the 'first-page' (for formatting
% reasons) and the remaining two appear in the \additionalauthors section.
%
\author{
% You can go ahead and credit any number of authors here,
% e.g. one 'row of three' or two rows (consisting of one row of three
% and a second row of one, two or three).
%
% The command \alignauthor (no curly braces needed) should
% precede each author name, affiliation/snail-mail address and
% e-mail address. Additionally, tag each line of
% affiliation/address with \affaddr, and tag the
% e-mail address with \email.
%
% 1st. author
\alignauthor
Chase Geigle\\
       \affaddr{Department of Computer Science}\\
       \affaddr{University of Illinois at Urbana-Champaign}\\
       \affaddr{Urbana, Illinois, USA}\\
       \email{geigle1@illinois.edu}
% 2nd. author
\alignauthor
ChengXiang Zhai\\
       \affaddr{Department of Computer Science}\\
       \affaddr{University of Illinois at Urbana-Champaign}\\
       \affaddr{Urbana, Illinois, USA}\\
       \email{czhai@illinois.edu}
}
% There's nothing stopping you putting the seventh, eighth, etc.
% author on the opening page (as the 'third row') but we ask,
% for aesthetic reasons that you place these 'additional authors'
% in the \additional authors block, viz.
%\additionalauthors{Additional authors: John Smith (The Th{\o}rv{\"a}ld Group,
%email: {\texttt{jsmith@affiliation.org}}) and Julius P.~Kumquat
%(The Kumquat Consortium, email: {\texttt{jpkumquat@consortium.net}}).}
%\date{30 July 1999}
% Just remember to make sure that the TOTAL number of authors
% is the number that will appear on the first page PLUS the
% number that will appear in the \additionalauthors section.

\maketitle
\begin{abstract}
  Massive open online courses (MOOCs) provide educators with an abundance
  of data describing how students interact with the platform, but this data
  is highly underutilized today. This is in part due to the lack of
  sophisticated tools to provide interpretable and actionable summaries of
  huge amounts of MOOC activity present in log data. To address this problem, we
  propose a student behavior representation method alongside a method for
  automatically discovering those student behavior patterns by leveraging
  the click log data that can be obtained from the MOOC platform itself.
  Specifically, we propose the use of a two-layer hidden Markov model
  (2L-HMM) to extract our desired behavior representation, and show that
  patterns extracted by such a 2L-HMM are interpretable, meaningful, and
  unique. We demonstrate that features extracted from a trained 2L-HMM can
  be shown to correlate with educational outcomes.
\end{abstract}

\newcommand{\NSFGRFP}{This material is based upon work supported by the
National Science Foundation Graduate Research Fellowship Program under
Grant Number DGE-1144245.}

\section*{Acknowledgments}
\NSFGRFP{}

\end{document}
