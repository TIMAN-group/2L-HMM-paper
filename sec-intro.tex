\section{Introduction}

The proliferation of massive open online courses (MOOCs) has resulted in a
profound impact on education. As more and more students participate in
these novel educational environments, it is of utmost importance that we be
able to understand the behavioral patterns students exhibit in these
environments. While we can easily observe the changes in behavior of
students in real classrooms, MOOC environments present some significant
challenges in this regard: the structure of the course itself is more
hands-off in nature than that of the traditional classroom (in most cases),
and thus attracts more students that are full-time workers with irregular
learning schedules.

At the same time, this influx of learners turning to MOOC platforms to
educate themselves directly leads to the collection of larger datasets of
behavioral data through the platform's logging. This presents a unique
opportunity: the data present in these logs has the power to aid us in
understanding the behavior of students who take our MOOCs. However, due to
the vast scale of these behavioral logs, student behavior patterns are
mostly undetectable for MOOC instructors and as a result the rich data
available through MOOC logs is highly underutilized today.

What stands in the way? Instructors require intelligent systems to create
concise and digestible summaries of the massive amount of interaction data
collected. If we can understand how users are interacting with our MOOCs,
we are much more likely to be able to make changes to these courses that
positively impact learners. We view this paper as attempting to bridge this
gap.

How should we represent behavioral patterns, and what does it mean to
understand changes in student behavior with respect to these patterns?
These are still very open questions, and are active areas of
research~\cite{Kizilcec:2013:LAK, Faucon:2016:EDM, Davis:2016:EDM,
Shih:2010:EDM}. In this paper, we advocate for a particular representation
of student behavior patterns as well as \emph{behavior transitions} that we
believe is simultaneously interpretable but also amenable to unsupervised,
automated discovery via statistical means. Specifically, we choose to
visualize behavior patterns as labeled directed graphs where a node
represents a ``behavior state'' (such as watching a lecture video or
visiting a forum), a directed edge indicates a transition from one
behavior state to another, node sizes are proportional to steady-state
probabilities, and edge widths are proportional to the probability of
leaving a node following that edge. We can use this same representation for
visualizing both the student behavior patterns as well as the transitioning
behavior between them. In Figure~\ref{fig:motivating-example} we show a
hypothetical example of the kind of output our proposed representation
could convey.  Here we see two different behavioral patterns
(\ref{fig:motivating-example-0} and \ref{fig:motivating-example-1}) as well
as the \emph{transition behavior} between these two behavior patterns
(\ref{fig:motivating-example-trans}).
\begin{figure}
  \centering
  \begin{subfigure}[t]{0.30\textwidth}
    \includegraphics[width=\textwidth]{figures/example/state0.png}
    \caption{Behavior pattern 0\label{fig:motivating-example-0}}
  \end{subfigure}
  \begin{subfigure}[t]{0.30\textwidth}
    \includegraphics[width=\textwidth]{figures/example/state1.png}
    \caption{Behavior pattern 1\label{fig:motivating-example-1}}
  \end{subfigure}
  \begin{subfigure}[t]{0.30\textwidth}
    \includegraphics[width=\textwidth]{figures/example/trans.png}
    \caption{Transitions between the two behavior patterns to the
    left.\label{fig:motivating-example-trans}}
  \end{subfigure}
  \caption{An idealized example of what our behavior representation could capture.}
  \label{fig:motivating-example}
\end{figure}
Such a visualized state-transition representation is very informative for
describing student behavior. Indeed, we could infer many things from even
such a simple example: The first might be that, when students are taking
quizzes, they tend to either use the forum or the videos for support, but
not both. They also tend to take quizzes in a sort of ``cycle'' pattern,
indicating perhaps that this course allows quiz re-takes. Finally, in
Figure~\ref{fig:motivating-example-trans} we could conclude that users tend
to change their quiz-taking behavior over time from one that is more
video-focused (pattern 0) to one that is more forum-focused (pattern 1).

Our goal in this paper is to design a model that can automatically capture
student behavior in this way via unsupervised learning methods applied to
large collections of click logs associated with MOOCs. We view our model
as a component of a system that enables collaboration between the machine
and a human instructor to extract knowledge from large collections of
MOOC data. Automatically extracting interpretable patterns from the
clickstream data associated with MOOCs is a necessary step in order for
instructors to identify the hidden knowledge in massive interaction
datasets. Without the availability of a suitable model for identifying
behavioral patterns, instructors are not empowered to use this available
data to improve their courses without expending extraordinary amounts of
manual effort (even with which the raw data can still be very hard to
interpret).

Our proposed model (as well as the proposed behavior representation) is
motivated by the following observations:
\begin{enumerate}
  \item Student behavior is complicated and cannot necessarily be captured
      sufficiently by rule-based methods such as those explored by
      \citet{Kizilcec:2013:LAK} and \citet{Davis:2016:EDM}. We instead
      propose to treat student behavior patterns as being characterized
      (represented) via a sequence of \emph{latent states}. This allows us
      to automatically capture patterns that we might not have been able to
      articulate clearly a priori via a series of rules, and also allows us
      to model the inherent uncertainty in assigning a student's behavior
      to a particular pattern or group.
  \item Student behavior can vary over time. Previous models that treat students
      as exhibiting only one behavioral pattern over
      time~\cite{Faucon:2016:EDM} miss out on the opportunity to understand
      student behavior dynamics in a course. We propose a latent space
      model with {\em latent state transitions} to flexibly model the
      dynamics.
  \item Analysis of student behavior can and should be performed at varying
      levels of granularity. This requires us to aggregate data over time
      with \emph{different levels of resolution}; existing models tend to come
      with a particular assumption about the resolution of time they
      consider~\cite{Faucon:2016:EDM, Kizilcec:2013:LAK, Shih:2010:EDM}. We
      propose a model that is agnostic to the time resolution considered,
      allowing it to be applied at different levels of resolution more
      naturally.
\end{enumerate}

Thus, what we propose is a \emph{latent variable approach} to mining student behavior
patterns that is \emph{probabilistic} for inference and
does \emph{not} force assumptions about time resolutions, making it
\emph{flexible to model state changes over different time resolutions} more
easily.  More specifically, we propose a novel two-layer hidden Markov
model (2L-HMM) to discover latent student behavior patterns via
unsupervised learning on large collections of student behavior observation
sequences.  Evaluation results on a MOOC data set on Coursera demonstrate
that the 2L-HMM can effectively discover a variety of interesting interpretable
student behavior patterns at different levels of resolution, many of which
are beyond what existing approaches can discover. We show that the patterns
uncovered by the 2L-HMM capture meaningful behavior by quantitatively
showing that features extracted from a trained 2L-HMM correlate with
learning outcomes. Since our proposed methods are
unsupervised, they can potentially be applied to any MOOC data without
requiring manual annotation effort at the level of sequences. Instead,
instructors are empowered to use the latent patterns that the 2L-HMM can
discover from raw data to extract knowledge about the behaviors his/her
students exhibit in the MOOC.
