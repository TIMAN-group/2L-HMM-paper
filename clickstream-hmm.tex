%% SIGCHI Proceedings Format (modern) Sample Document
%% This sample file demonstrates the use of the `sigchi-modern' class file.
%%
%% The content of this document draws heavily on the `HCI Archive Format'
%% template supplied by SIGCHI, but has been modified to demonstrate and
%% document the LaTeX class.

% The parameter given to the `\documentclass' command can be one of:
%   `preprint', `submission', or `final'.
% These control the presence of certain features for each stage of the authoring
% process. For example, `submission' does not output the author block, and
% `final' disables page numbers.
%
% The \preprintonly{}, \submissiononly{}, and \finalonly{} commands allow you
% to wrap text that will only be output under the corresponding option.

\documentclass[a4,notitlepage,12pt]{jedm}

\def\meta/{\textsc{MeTA}}

\newcommand{\citemeta}{\cite{Massung:2016:ACL}}
\usepackage{graphicx}
\usepackage{booktabs}

\usepackage{microtype}

\usepackage{natbib}

\usepackage{subcaption}
\usepackage[hidelinks]{hyperref}
\usepackage[inline]{enumitem}
\usepackage{amsmath}
\usepackage{pdflscape}
\usepackage{afterpage}
\usepackage{microtype}

\setlength{\bibhang}{0em}

\renewcommand\cite{\citep}

\begin{document}

\title{Modeling MOOC Student Behavior With Two-Layer Hidden Markov Models}

\author{%
  {\large Chase Geigle}\\Department of Computer Science\\University
  of Illinois at Urbana-Champaign\\Urbana, Illinois,
  USA\\geigle1@illinois.edu
  \and
  {\large ChengXiang Zhai}\\Department of Computer Science\\University of
  Illinois at Urbana-Champaign\\Urbana, Illinois, USA\\czhai@illinois.edu}

\maketitle

\begin{abstract}
  Massive open online courses (MOOCs) provide educators with an abundance
  of data describing how students interact with the platform, but this data
  is highly underutilized today. This is in part due to the lack of
  sophisticated tools to provide interpretable and actionable summaries of
  huge amounts of MOOC activity present in log data. To address this problem, we
  propose a student behavior representation method alongside a method for
  automatically discovering those student behavior patterns by leveraging
  the click log data that can be obtained from the MOOC platform itself.
  Specifically, we propose the use of a two-layer hidden Markov model
  (2L-HMM) to extract our desired behavior representation, and show that
  patterns extracted by such a 2L-HMM are interpretable and meaningful. We
  demonstrate that the proposed 2L-HMM can also be used to extract latent
  features from student behavioral data that correlate with educational
  outcomes.
\end{abstract}

\newcommand{\NSFGRFP}{This material is based upon work supported by the
National Science Foundation Graduate Research Fellowship Program under
Grant Number DGE-1144245 and by the National Science Foundation Research
Program under Grant Number IIS-1629161.}

\newcommand{\acknowledgments}{The authors would like to thank Hao Zheng,
Jason Cho and Sean Massung for their helpful discussion while developing
this work, and the ATLAS group at UIUC for their help with securing the
datasets used. \NSFGRFP{}}

\section{Introduction}
The proliferation of massive open online courses (MOOCs) has resulted in a
profound impact on education. As more and more learners turn to MOOCs to
educate themselves on various topics, more and more behavioral data is
being collected as part of the system on which the MOOC is offered. The
data present in these logs has the power to aid us in understanding the
behavior of students who take our MOOCs, which is mostly undetectable for
instructors of these MOOCs today due to its vast scale. As a result, the
rich data available through these MOOC logs is highly underutilized today.

What stands in the way? Clearly, intelligent systems to create concise and
digestible summaries of the massive amount of interaction data collected
are needed in order to empower the instructors of these courses. If we can
understand how users are interacting with our MOOCs, we are much more
likely to be able to make changes to these courses that positively impact
learners. While we can easily observe the changes in behavior of students
in real classrooms, MOOCs present a challenge due to their hands-off
nature and sometimes irregular schedule due to being a full-time worker.   
We view this paper as attempting to bridge this gap. Specifically,
in this paper we propose unsupervised learning methods for automatically
discovering and characterizing student learning behavior patterns or profiles from large
collections of click logs associated with MOOCs. 

Our work is motivated by the following observations:
\begin{enumerate}[itemsep=2pt,topsep=2pt]
  \item Student behavior is complicated and cannot necessarily be captured
      sufficiently by rule-based methods such as those explored by
      \citet{Kizilcec:2013:LAK}. We instead propose to treat
      student behavior patterns as being characterized (represented) via a sequence of
      \emph{latent states}. This allows us to automatically capture
      patterns that we might not have been able to articulate clearly a
      priori via a series of rules, and also allows us to model the
      inherent uncertainty in assigning a student's behavior to a
      particular pattern or group. 
  \item Student behavior can vary over time. Previous models that treat students
      as exhibiting only one behavioral pattern over
      time~\cite{Faucon:2016:EDM} miss out on the opportunity to understand
      student behavior dynamics in a course. We propose a latent space model to flexibly model the dynamics. 
  \item Analysis of student behavior can and should be performed at varying
      levels of granularity. This requires us to aggregate data over time
      with \emph{different levels of resolution}; existing models tend to come
      with a particular assumption about the resolution of time they
      consider~\cite{Faucon:2016:EDM, Kizilcec:2013:LAK, Shih:2010:EDM}. We propose a more flexible model to accommodate different levels of resolution. 
\end{enumerate}

Thus, what we propose is a \emph{latent variable approach} to mining student behavior
patterns that is \emph{probabilistic} for inference and 
\emph{flexible to model state changes over different time resolutions}. 
More specifically, we propose a novel two-layer hidden Markov model (TL-HMM) to
 discover latent student behavior patterns via
unsupervised learning on large collections of student behavior observation
sequences. Evaluation results on a MOOC data set on Coursera demonstrate that TL-HMM can effectively 
discover a variety of interesting interpretable student behavior patterns at different levels of resolution, many of which are beyond what existing approaches can discover. TL-HMM further enables easy use of the discovered patterns for discriminative analysis such as prediction of learning outcome. 
Since our proposed methods are unsupervised, they can potentially be applied to any MOOC data without requiring any manual work to facilitate understanding of student behaviors and their variations, opening up many possibilities for developing intelligent tutoring systems that can adaptive to student behavior. 




\section{Related Work}

Our model is based heavily upon the prior art of Hidden Markov
Models~\cite{Rabiner:1990:RSR} for sequence labeling tasks. These types of
models are widely applicable and have been used for tasks such as speech
recognition~\cite{Huang:1990:HMM}, part-of-speech
tagging~\cite{Jurafsky:2009:SLP}, and econometrics~\cite{Hamilton:1990:JoE}
and are a member of the more general family of probabilistic graphical
models~\cite{Koller:2009:PGM}.  A major challenge in applying HMMs and
other graphical models successfully to solve a problem is to design an
appropriate architecture of the model, which always varies according to
specific applications.

A multi-layered approach to HMM modeling of sequence data has been
performed before in other domains. \citet{Zhang:2004:CVPR}, for example,
explored a two-layer HMM framework for modeling actions in meetings, but
their definition of ``two-layer'' differs from ours. In their formulation,
the ``lower-layer'' level is used to label audio-video action sequences
into basic events, and the ``upper-layer'' is used to label the output of
the lower-layer to discover higher-level office behavior abstractions. Our
formulation differs in that we do not feed the labeled sequence of the
lower level into the input of the higher level. Instead, our lower level is
actually treated using a non-hidden Markov model, and the higher level is
modeling transitions between the $K$ different non-hidden Markov models we
consider.

Our formulation is more closely related to the Hierarchical Hidden Markov
Model (HHMM) detailed in \citet{Fine:1998:ML}. Here, the ``layers'' are
modeled by having the hidden Markov model have two kinds of transitions.
Horizontal transitions move between states within a layer, where vertical
transitions move between different layers. At the bottom layer lie the
``production'' states, which output symbols according to some probability
distribution. Our specific model in this case can be modeled as a HHMM
where the horizontal transitions between nodes at the highest layer
(including self-loops) \emph{must} be immediately followed by a vertical
transition to the lower layer. The output probability distributions over
symbols in the lower layer ``production'' level are forced to emit only one
kind of symbol, and vertical transitions are only allowed into the original
higher-layer state that transitioned down into the lower-layer.

Mixtures of hidden Markov models are also conceptually similar to our
formulation. \citet{Song:2009:NDSS} explored using a mixture of hidden
Markov models in the context of anomaly detection in the security domain.
\citet{Ypma:2002:Springer} use mixtures of HMMs to categorize web pages and
cluster users by investigating web log data, which is quite similar to the
clickstream log data we obtain from MOOCs. The major difference between
our approach and a standard mixture of HMMs is that we also model the
\emph{transition behavior} between the Markov models that make up our
model's lowest layer, where a standard mixture of HMMs would ignore the
potential dependence of the previous sequence's latent state on the next
sequence's latent state.

HMMs or similar ideas have been previously applied to model education
data~\cite{Shih:2010:EDM,Kizilcec:2013:LAK,Davis:2016:EDM}, but the
previous models are not well tuned toward the student behavior task and
thus cannot adequately address all the aspects of complexity of student
learning behavior.  A main technical contribution of this paper is to
propose a more general HMM that can better adapt to the variations of
student behavior via its variable resolution and nested HMM structure, and
thus enable discovery of more sophisticated behavior patterns and provide
more detailed characterization of student behavior than the previous
models.

For example, \citet{Kizilcec:2013:LAK} assigned students to states
following a rule-based approach based upon when the student submitted the
assignment for a particular week in the course. They investigated how
students transitioned between these states as the course progressed, and
used the sequence of states a student exhibited as a representation for
performing $k$-means clustering of students into related groups. This
differs from our method substantially: we assign students to states using a
probabilistic framework to account for uncertainty in this state
assignment, and jointly learn \emph{representations} for these states,
which are treated as being \emph{latent} as opposed to pre-defined using
some rule (or set of rules).  Furthermore, our model provides more
flexibility in how the time segments are defined, allowing for both finer
(for example, day-by-day) or coarser (for example, month-by-month)
granularity. \citet{Shih:2010:EDM} investigated the use of HMM-based
clustering techniques for automatic discovery of student learning
strategies when solving a particular problem.  This is similar to our
approach in that the description of behavior profiles is a Markov model,
but cannot further characterize each latent state with another informative
HMM. Thus their work can be regarded to modeling ``micro'' behavior,
whereas our model can model both ``micro'' and ``macro'' behavior.

\citet{Davis:2016:EDM} investigate frequent student behavior pattern chains
with a set of actions that is defined similarly to ours. However, their
method for finding the common behavioral patterns involves a manual
clustering step to identify behavioral ``motifs,'' which is then followed
by an automatic (rule-based) assignment of all sequences to these motifs.
Our method, by contrast, attempts to do this automatically: the latent
state representations obtained by our model attempt to capture similar
meanings to their behavior motifs in a completely automated fashion. They
also automatically generate and investigate Markov models for different
MOOCs, but do so by considering \emph{all} student action sequences as
belonging to a \emph{single Markov model}. In our approach, we allow each
student behavior sequence to belong to one of $K$ different Markov models
(and further model the transition probabilities between these latent state
Markov models between each sequence a student generates). Thus, their
Markov models presented are a special case of our model when $K = 1$.

\citet{Faucon:2016:EDM} proposed a semi-Markov model for simulating MOOC
students. They produce behavior profiles that characterize groups of
students in the form of semi-Markov models like our proposed model does,
but they assume that a student can belong to only one behavior profile
across the entire course rather than allowing this profile to change over
time. Because we do not have this restriction, our model is also able to
learn the transition probabilities between the different behavior profiles
we discover.

There are a few additional related studies worth mentioning.
Bayesian Knowledge Tracing~\cite{Corbett:1994:UMUAI} in its basic form uses
a hidden Markov model to model the probability that a learner knows a
certain skill at a given time. Modifications to this algorithm include
contextual estimation of the ``slip'' and ``guess'' probabilities of the
model~\cite{Baker:2008:ITS} and most recently a re-framing as a neural
network problem~\cite{Piech:2015:NIPS}.



\section{A Two-Layer HMM for MOOC Log Analysis}

Our general idea is to use a probabilistic generative model to model 
the student activities as recorded in a MOOC log, which means we will 
assume that all the observed student activities are samples drawn (i.e., ``generated'') from 
a parameterized probabilistic model. We can then estimate the 
parameter values of the probabilistic model by fitting the model to a specific
MOOC log data set. The estimated parameter values could then be treated as 
the latent ``knowledge'' discovered from the data. Because such a generative model 
attempts to fit {\em all} the data, it enables us to discover interesting patterns
that can explain the {\em overall} behavior of a student or the {\em common} behavior patterns shared by many students. 

An HMM  is a specific probabilistic generative model with a ``built-in'' state transition system
that would control the data to be generated by the model, thus it is especially 
suitable for modeling sequence data~\cite{Rabiner:1990:RSR}. At any moment, the HMM would be in 
one of $k$ states $U=\{u_1, ...,u_k\}$, and at the next moment, the HMM would move to 
another state stochastically according to a transition matrix that specifies the probability of
going to state $u_i$ when the HMM is currently in state $u_j$, i.e., $p(u_i|u_j)$. 
When the HMM is in state $u$, the HMM can generate an observable data point $x$ 
according to an output probabilistic model $p(x|u)$. Thus if we ``run'' an HMM for 
$N$ time points denoted by $t=1, ..., N$, the HMM could ``generate'' a sequence of 
observations $x_1 ... x_N$, where each $x_i$ is an output symbol by 
going through a sequence of {\em hidden states} $w_1 ... w_N$ where $w_i \in U$ is a state. 
The association of such a latent sequence of state transitions with the observed symbols makes
it possible to use HMM to ``decode'' the latent behavior of students behind the surface behavior we directly observe in the log data, allowing for understanding student behavior more deeply than a model with no latent state variables. 

In many ways, the generation process behind an HMM is meant to simulate the
actual behavior of a student. We assume that students transition through different ``task states'' (or ``behavior states'')  in the process of study. 
One such task state may be to learn about a topic by mostly watching
lecture videos, another task state may be to work on quizzes, and yet
another may be to participate in forum discussions. While in each of these
different states, the student would tend to exhibit different surface
``micro'' behaviors. For example, in the lecture study state, the student
would tend to have many video-watching related behaviors and occasionally
forum activities, while in the quiz-taking state (in order to pass each
module), the student would tend to show many quiz-related ``micro''
activities as well as asking questions or checking discussions on the
forum. Note that due to the complexity of the student behavior, it is very
difficult to accurately {\em prescribe} the specific surface ``micro''
behavior patterns for each state in advance, especially without  prior knowledge about the students. For example, forum activities are likely interleaved with other activities   in every task state and the interleaving pattern can be somewhat irregular with potentially many variations. 
The major motivations for using an HMM are that 
\begin{enumerate*}[label=(\arabic*)]
    \item it uses a probabilistic model (i.e., the output probability distribution $p(x|u)$ conditioned on each state) to directly capture the inevitable uncertainty in the association of surface ``micro'' activities with their corresponding latent task/behavior state, which is often our main target to discover and characterize, and 
    \item it does not make any assumption about which latent task/behavior
      state must be associated with which observed activities or how a
      student would move from one state to another, but instead allows our
      data to ``tell'' us what kind of associations are most likely,
      what kind of transitions are most probable, and which states tend to be more long-lasting for any particular set of students. 
\end{enumerate*}

However, if we use an ordinary HMM to analyze our data, we would treat each
observed ``micro'' activity (e.g., video watching, or forum post reading)
as an output symbol, and thus the output distribution $p(x|u)$ for each
discovered latent state would be a simple distribution over all kinds of
observable micro activities recorded in our log data (e.g., 50\% lecture
watching, 8\% quiz taking, 7\% quiz submission, 2\% course wiki reading,
...). While such a distribution is meaningful and can already help us
interpret the corresponding latent state, it only gives us a rather
superficial characterization of student behavior.

Ideally, we want $p(x|u)$ to characterize the directly observable ``micro''
behavior in more detail to further capture the relations and dependencies
of these micro activities. To this end, we would treat an {\em entire
sequence} of micro activities (e.g., one session of activities) as an
observed ``symbol'' from a latent state, and further model the generation
of such a sequence with another Markov model where we treat each micro
activity as an {\em observable} state, and model the transitions between
these activity states in very much the same way as the state transitions in
HMM.

Adding this second layer would allow us to characterize a latent task
state in much more detail, as it would reveal not only what activities are
most common to a task state, but also the transition patterns between these
``micro'' activities (e.g., it can reveal frequent back-and-forth
transitions between quiz-taking and quiz-submission, which would suggest a
concentrated period of taking quizzes). Combining this ``surface'' Markov
model over the ``micro'' actions with the ``deep'' hidden Markov model over
the latent task states gives us a fairly general and powerful two-layer HMM
(TL-HMM) that can simultaneously learn ``deeply'' the latent task/behavior
states and their transitions as well as the corresponding ``micro''
activity transition patterns associated with each latent state to
facilitate interpretation and analysis of the discovered latent state
patterns. Our implementation of the learning algorithm for TL-HMMs is
included as part of the \meta/ toolkit~\citemeta{}.


\newcommand{\textretrieval}{\preprintonly{textretrieval-001}%
    \finalonly{textretrieval-001}%
    \submissiononly{moocname1-00X}}

\newcommand{\sustain}{\preprintonly{sustain-001}%
    \finalonly{sustain-001}%
    \submissiononly{moocname2-00X}}

\newcommand{\UIUC}{\preprintonly{UIUC}%
    \finalonly{UIUC}%
    \submissiononly{(redacted University)}}

\section{Results}
To qualitatively evaluate our model, we look at the latent state
representations we learn by fitting the model to the log data associated
with two different Coursera MOOCs: \textretrieval and \sustain.


\section{Limitations and Potential Drawbacks}

There are a few limitations of our model that are important to highlight.
First, there are specific technical limitations due to the statistical
nature of the model and the particular methodology we propose for fitting
our model parameters. Second, there are limitations in the kinds of
patterns our model is able to discover in its current formulation, and the
ease with which instructors are able to extract knowledge from these
patterns. We discuss both of these lines below.

\subsection{Technical Limitations and Implementation Challenges}

 %irregular behavior
%patterns In particular, because
One potential limitation is that the model is complex and has many parameters,
to truly uncover relevant patterns in the data, thus a large amount of it must
be available to the training algorithm. The assumption that we have a large
amount of sequence data available for training on should generally hold for
most MOOC courses, but this assumption may be problematic if attempting to
apply our model to smaller online (or on-campus) courses.

Our model fits its parameters using maximum likelihood estimation using the
EM algorithm. The EM algorithm is a hill climbing algorithm that is
optimizing in a highly non-convex parameter space. Thus, it can only
guarantee reaching a local maximum in practice~\citep{Dempster:1977:JRSS}.
This may mean that the parameters found for the model may not be the
``best'' parameters in a global sense, which may lead to suboptimal latent
state representations and transition patterns. The problem may not be as serious as it 
appears to be because strong patterns tend to always show up as long as the algorithm reaches a reasonably good local maximum, and the differences of the results tend to be 
related to weak ``unstable'' patterns which may not be reliable anyway. 
 A commonly applied approach
to address the problem of multiple local maxima is to run the model multiple times with different
starting points to allow the model to explore a larger portion of the
potential parameter space. One can then compare the log-likelihood of the
data between the models that were started from different initialization
points and select the one that has the highest value. This still does not
guarantee finding a global maximum, but it does help alleviate the
potential for finding a particularly bad local maximum.  
In practice, we
have found that our model can be fit to the data fairly quickly on
commodity hardware, so running it multiple times to address this concern is
not computationally unfeasible. 


A further complication in blindly applying the EM algorithm to our model is
the fact that the observation probabilities will be incredibly small. The
probability that a specific sequence is generated by a specific Markov
chain (one of our latent states) will decrease exponentially with its
length. While there are general approaches to avoiding numerical underflow
in hidden Markov models, applying the ``scaling'' method proposed by
\citet{Rabiner:1990:RSR} will still result in numeric stability issues in
our case due to the incredibly small sequence-generation probabilities. We
address this in our open-source implementation by computing the trellises
in log-space and using the log-sum-exp trick when we need to take
summations, which exploits the identity
\begin{equation}
    \log \sum_{i=1}^n e^{x_i} = a + \log \sum_{i=1}^n e^{x_i - a}
\end{equation}
where $a$ is typically set to $\max_i x_i$ to improve stability. We did not
encounter further stability issues once we applied these two tricks.

As is the case with traditional hidden Markov models, it is often important
to smooth the underlying model's distributions to ensure that there is a
non-zero probability of generating the observations. We employed a simple
additive smoothing in our implementation with a small additive constant for
all of our transition matrices to avoid this problem.

\subsection{Limitations of Discovered Patterns}

The proposed behavior representation is most suitable for representing
recurring behavior patterns, which presumably are most interesting 
to extract from the data, but may not cover all the interesting patterns in the data. 
It would be interesting to further explore other complementary models such as
 time series models, which may help capture non-recurring patterns.  
 
Our proposed model is flexible in the patterns it can discover in two main
ways. The first is that the granularity of the patterns can be adjusted by
changing the segmentation strategy one uses to divide the user action
stream into discrete ``sessions''. The other is in the number of latent
states $K$ that are used to describe the segmented action sequences. One
drawback of these two lines of flexibility is that there is not necessarily
a clear answer as to the ``correct'' approach for both in any given
scenario. Varying the segmentation strategy changes the granularity of the
patterns that are required to be explained by the latent states, which will
change their meaning. Varying the number of latent states increases the
flexibility of the model, but also may result in latent states that are not
substantially different from the other latent states and/or latent states
with very low probability. This flexibility forces a user of our model to
make some assumptions about the patterns they wish to find (granularity,
diversity), and the model itself does not necessarily provide clear
guidance as to what the best approach is.

Furthermore, we have made an implicit assumption that the segmentation
strategy is applied as a pre-processing step (and is most obviously a
deterministic process). The proposed segmentation strategies in this paper
do not specifically allow for the transitioning between the different
latent behavior patterns to occur over different windows of time for
different users: we have made a strong assumption that transitions between
latent states only occur at action sequence boundaries. One can model how
much time a user stays in a given state in terms of the number of sessions
they remain there before transitioning, but it may be better to directly
model the amount of time we expect a user to stay in the given state
directly. In other words, a more powerful model might be one in which the
segmentation and the latent behavior pattern discovery are jointly modeled
in a single probabilistic framework rather than being completely separate
pieces as we investigate in this work.

While the model can discover \emph{patterns} in the data in a fully
automated way, there is still clearly a burden on the user of our model to
interpret the patterns it has discovered to create actionable
\emph{knowledge} about the MOOC from which the data was extracted.
Extracting the patterns is a necessary step towards the creation of
knowledge, and we view our model as a component in a collaborative system
which leverages the machine to perform statistical modeling to extract
patterns which then enable a human actor to extract knowledge and take
actions on the basis of the data. The pattern discovery is an important
and absolutely necessary component, without which it would be very difficult, if not impossible, 
for humans to directly digest the student behavior buried in the large amount of 
data.  Of course, pattern discovery is only a means to help humans obtain knowledge, not the end of the knowledge discovery process.


\section{Conclusions and Future Work}
% summarize the main contributions of the paper; summarize the experimental
% results; argue again for our novelty; discuss future directions
As a tool to help instructors and education researchers better understand
the behavior of MOOC students, we proposed a two-layer hidden Markov model
to automatically extract student activity patterns in the form of behavior
state-transition graphs from large amounts of MOOC log data. This model is
different from existing methods in that it treats behavior patterns as a
sequence of \emph{latent states}, rather than assigning these states in a
rule-based manner. It captures the variable behaviors of students over time
and allows analysis at different levels of granularity.

We showed that such a model does, in fact, capture meaningful behavior
patterns and produces descriptions of these behavior patterns that are easy
to interpret. We argued that it is important to capture student behavior
patterns with more sophisticated models than simple discrete distributions
over actions in order to capture information present in bigrams of actions
(or above). By varying the number of latent
states inferred, we showed that the model is flexible and can capture
patterns of differing levels of specificity in this way.  Finally, we
investigated whether we can detect differences in student behavior patterns
as they correlate with course performance. Specifically, we demonstrated
that high-performing students produce substantially different HMM
transition diagrams that tend to show longer concentration span in quiz-taking
and more active  forum participation as compared with the average students.
These results show the great potential of the proposed model for serving as a tool
to help humans discover knowledge about student behaviors.  

%\subsection{Future Work}

There are a number of interesting directions to further extend our work in the future. 
First, the proposed model is 
completely general and can thus be easily applied to analyze the log of many
other courses to enable deep understanding of student behaviors as well as
the correlations of such behaviors and other variables such as grades. To realize these benefits, 
it would be useful 
to develop a MOOC log analysis system based on the proposed model so as to facilitate education research and help instructors improve course
design.

Second, our model can empower many new comparative analyses. For example, we could
now look at how behavior patterns change between different offerings of
the same MOOC to understand how changes in course structure or materials
influence student behavior. Individual students can now also be compared
against each other or against groups. For example, by decoding the latent
state sequences for each student, we can measure how ``surprising'' their
latent state transition sequence is relative to the average we would expect
according to the model, or to the average ``perfect'' student, etc. We can
now investigate how certain behavioral patterns correlate with properties
of a student (e.g., demographics, prior aptitude, etc.). After decoding the
students' latent state sequences, we could also correlate course-wide
drifts in these latent states with events in a course. For example, we
might be able to automatically discover difficult or confusing parts of a
course by noticing spikes in the distribution of students over latent
states over time.

Third, there is more recent work on better learning algorithms for mixtures of
Markov models~\citep{Gupta:2016:NIPS}. It would be worth exploring whether
the advances proposed in this and similar work can be applied to our model
to address some of the concerns surrounding our use of the EM algorithm for
our parameter estimation.

Finally, the proposed model can be extended in several ways. For example, 
although our model does not explicitly model drop-out like
\citet{Kizilcec:2013:LAK}, doing so is an obvious extension. Our model
would be able to provide predictions of when a student is ``at risk'' for
dropping out under such an extension.
Also, currently, the model learns a transition matrix over the latent states that
is \emph{shared} across all students. It would be interesting to instead
learn a different latent state transition matrix for each individual
student, but keep the second-level Markov models shared. This would provide
the model with more flexibility which may be desirable itself, but would
also naturally result in a description of a student (via his or her HMM
transitions) that could be incorporated into existing supervised learning
techniques that try to predict student outcomes for understanding
which of the latent behavior patterns discovered by 2L-HMM are most predictive
of the performance of student learning. One could also relax this somewhat
and posit that \emph{groups} of students transition between the lower layer
patterns identified by our method in distinct ways; this way, we can do a soft
clustering of students into $K_2$ groups based on the similarity of their
transitioning behavior between the higher-level behaviors we can identify.




\section{Acknowledgments}
\acknowledgments{}

\bibliographystyle{acmtrans}
\bibliography{bib}
\end{document}
