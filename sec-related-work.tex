\section{Related Work}

\citet{Faucon:2016:EDM} propose a semi-Markov model for simulating MOOC
students. They produce behavior profiles that characterize groups of
students in the form of semi-Markov models like our proposed model does,
but they differ from our work in two major ways. First, they assume that a
student can belong to only one behavior profile across the entire course
rather than allowing this profile to change over time.  Because we do not
have this restriction, our model is also able to learn the transition
probabilities between the different behavior profiles we discover.

\citet{Shih:2010:EDM} investigate the use of HMM-based clustering
techniques for automatic discovery of student learning strategies when
solving a particular problem. This is similar to our approach in that the
description of behavior profiles is a Markov-model, but differs in focus:
we investigate macro-level MOOC behavior here as opposed to looking at
individual problem solving behavior. \citet{Ypma:2002:Springer} use
mixtures of HMMs to categorize web pages and cluster users by investigating
web log data, which is quite similar to the clickstream log data we obtain
from MOOCs.

\citet{Kizilcec:2013:LAK} assign students  to states following a rule-based
approach based upon when the student submitted the assignment for a
particular week in the course. They investigated how students transitioned
between these states as the course progressed, and used the sequence of
states a student exhibited as a representation for performing $k$-means
clustering of students into related groups. This differs from our method
substantially: we assign students to states using a probabilistic
framework to account for uncertainty in this state assignment, and jointly
learn representations for these states, which are treated as being
\emph{latent} as opposed to pre-definied using some rule (or set of rules).
Furthermore, our model provides more flexibility in how the time segments
are defined, allowing for both finer (for example, day-by-day) or courser
(for example, month-by-month) granularity.
