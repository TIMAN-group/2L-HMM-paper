\section{Conclusions and Future Work}
% summarize the main contributions of the paper; summarize the experimental
% results; argue again for our novelty; discuss future directions
As a tool to help instructors and education researchers better understand
the behavior of MOOC students, we proposed a two-layer hidden Markov model
to automatically extract student activity patterns in the form of behavior
state-transition graphs from large amounts of MOOC log data. This model is
different from existing methods in that it treats behavior patterns as a
sequence of \emph{latent states}, rather than assigning these states in a
rule-based manner. It captures the variable behaviors of students over time
and allows analysis at different levels of granularity.

We showed that such a model does, in fact, capture meaningful behavior
patterns and produces descriptions of these behavior patterns that are easy
to interpret. We argued that it is important to capture student behavior
patterns with more sophisticated models than simple discrete distributions
over actions in order to capture information present in bigrams of actions
(or above). By varying the number of latent
states inferred, we showed that the model is flexible and can capture
patterns of differing levels of specificity in this way.  Finally, we
investigated whether we can detect differences in student behavior patterns
as they correlate with course performance. Specifically, we demonstrated
that high-performing students produce substantially different HMM
transition diagrams that tend to show longer concentration span in quiz-taking
and more active  forum participation as compared with the average students.
These results show the great potential of the proposed model for serving as a tool
to help humans discover knowledge about student behaviors.  

%\subsection{Future Work}

There are a number of interesting directions to further extend our work in the future. 
First, the proposed model is 
completely general and can thus be easily applied to analyze the log of many
other courses to enable deep understanding of student behaviors as well as
the correlations of such behaviors and other variables such as grades. To realize these benefits, 
it would be useful 
to develop a MOOC log analysis system based on the proposed model so as to facilitate education research and help instructors improve course
design.

Second, our model can empower many new comparative analyses. For example, we could
now look at how behavior patterns change between different offerings of
the same MOOC to understand how changes in course structure or materials
influence student behavior. Individual students can now also be compared
against each other or against groups. For example, by decoding the latent
state sequences for each student, we can measure how ``surprising'' their
latent state transition sequence is relative to the average we would expect
according to the model, or to the average ``perfect'' student, etc. We can
now investigate how certain behavioral patterns correlate with properties
of a student (e.g., demographics, prior aptitude, etc.). After decoding the
students' latent state sequences, we could also correlate course-wide
drifts in these latent states with events in a course. For example, we
might be able to automatically discover difficult or confusing parts of a
course by noticing spikes in the distribution of students over latent
states over time.

Third, there is more recent work on better learning algorithms for mixtures of
Markov models~\citep{Gupta:2016:NIPS}. It would be worth exploring whether
the advances proposed in this and similar work can be applied to our model
to address some of the concerns surrounding our use of the EM algorithm for
our parameter estimation.

Finally, the proposed model can be extended in several ways. For example, 
although our model does not explicitly model drop-out like
\citet{Kizilcec:2013:LAK}, doing so is an obvious extension. Our model
would be able to provide predictions of when a student is ``at risk'' for
dropping out under such an extension.
Also, currently, the model learns a transition matrix over the latent states that
is \emph{shared} across all students. It would be interesting to instead
learn a different latent state transition matrix for each individual
student, but keep the second-level Markov models shared. This would provide
the model with more flexibility which may be desirable itself, but would
also naturally result in a description of a student (via his or her HMM
transitions) that could be incorporated into existing supervised learning
techniques that try to predict student outcomes for understanding
which of the latent behavior patterns discovered by 2L-HMM are most predictive
of the performance of student learning. One could also relax this somewhat
and posit that \emph{groups} of students transition between the lower layer
patterns identified by our method in distinct ways; this way, we can do a soft
clustering of students into $K_2$ groups based on the similarity of their
transitioning behavior between the higher-level behaviors we can identify.


