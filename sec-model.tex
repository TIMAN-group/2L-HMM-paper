\section{Model}

Given a MOOC log, we can define a set $\mathbf{A}$ of actions that a
student can take at any given time. For example, an action $a \in
\mathbf{A}$ might be ``viewing lecture'', ``taking quiz'', or ``viewing
forum''. For each student in the course $\ell \in \mathbf{L}$, we then
extract a list of action sequences $\mathbf{S}_\ell$ that he or she
produced as observed in the log, where each sequence $\mathbf{s} \in
\mathbf{S}_\ell$ is itself a list of actions $(a_1, a_2, \ldots, a_M)$ with
each $a_i \in \mathbf{A}$.  Each sequence can be divided flexibly; in this
paper we chose to denote the end of a sequence as occurring when no further
actions occur within a 10 hour window of time (and thus our sequences
roughly correspond to one day's worth of activity).

Each sequence $\mathbf{s} \in \mathbf{S}_\ell$ is associated with a latent
state $u \in \{1,\ldots,K\}$ (where $K$ is a fixed constant picked in
advance). The actions within the sequence $\mathbf{s} = (a_1, a_2, \ldots,
a_M)$ are then modeled as a first-order Markov process conditioned upon $u$
where each action is drawn from a distribution conditioned upon the
previous action (except for the first which is sampled from an initial
starting distribution). We can write the parameters for the first-order
Markov model associated with latent state $u$ as $\lambda^{(u)} =
(\pi^{(u)}, A^{(u)})$ where $\pi^{(u)}$ indicates the initial probability
vector of length $|\mathbf{A}|$ and $A^{(u)}$ is an $|\mathbf{A}| \times
|\mathbf{A}|$ matrix indicating the transition probabilities between each
pair of actions from $\mathbf{A}$.

Thus, the probability of a sequence $\mathbf{s}$ of length $M$ given its
latent state $u$ is
\begin{equation}
  P(\mathbf{s} \mid \lambda^{(u)}) = P(a_1 \mid \pi^{(u)}) \prod_{i=2}^M P(a_i \mid
  a_{i-1}, A^{(u)})
\end{equation}
where $P(a \mid \pi^{(u)}) = \pi^{(u)}_a$ is the probability of starting
with action $a$ and $P(a_i \mid a_{i-1}, A^{(u)}) = A^{(u)}_{a_{i-1}, a_i}$
is the transition probability of moving from action $a_{i-1}$ to $a_{i}$.

We can compute the likelihood of a list of action sequences
$\mathbf{S}_\ell$ of length $N$ for a student $\ell$ as
\begin{equation}
  P(\mathbf{S}_\ell \mid \Lambda)
  = \prod_{i=1}^N \sum_U P(U \mid \Lambda) P(\mathbf{s}_i \mid
  \lambda^{(u_i)})
\end{equation}
where $\Lambda$ is the set of all model parameters, and the summation over
$U$ indicates a sum over all possible latent state sequences of length
$|\mathbf{s}_i|$. In our model, we let $\Lambda = (\pi, A, \lambda^{(1)},
\ldots, \lambda^{(K)})$ where $\pi$ and $A$ are the parameters of a first-order
Markov model over the latent states and each $\lambda^{(i)}$ consists of
the parameters for the first-order Markov model over action sequences for
latent state $i$. Thus $\pi$ (without superscripts) is an initial
probability vector of length $K$ and $A$ (without superscripts) is a $K
\times K$ transition probability matrix, analogous to the case with the
individual first-order Markov model parameters $\lambda^{(i)}$ for each
latent state.

This can be seen as a modification of the traditional hidden Markov
model~\cite{Rabiner:1990:RSR} where instead of discrete observations (one
for each latent state transition) we have observations that take the form
of entire sequences $\mathbf{s}_i = (a_1, a_2, \ldots a_M)$ whose
probabilities are computed using another (non-hidden) Markov model
conditioned upon the latent state $u_i$.

\subsection{Parameter Estimation}
To learn the parameters of our model, we may attempt to use maximum
likelihood estimation. Unfortunately, a closed-form solution does not
exist, so we must appeal to the EM algorithm~\cite{Dempster:1977:JRSS}. In
particular, we propose a minor modification of the Baum-Welch
algorithm~\cite{Rabiner:1990:RSR} which is used to learn the parameters for
hidden Markov models. In the following sections, we will provide a brief
description of the original Baum-Welch algorithm, and then describe our
modification.

\subsubsection{Baum-Welch for Traditional HMMs}
In the traditional HMM formulation, we have $\Lambda = (\pi, A, B)$ where
$\pi$ is the initial probability distribution over the latent states,
$A$ is a $N\times N$ matrix indicating the latent state transition
probabilities, and $B$ is a $N \times V$ matrix indicating the probability
of generating a discrete observation symbol from a given latent state. The
Baum-Welch algorithm (also called the forward-backward algorithm) defines
two sets of variables $\alpha_t(i)$ called the forward variables and
$\beta_t(i)$ called the backward variables.

$\alpha_t(i) = P(o_1, \ldots, o_t, q_t = i \mid \Lambda)$ is the probability of
generating the sequence of observations $(o_1, o_2, \ldots, o_t)$ up to time
$t$ and arriving in state $i$ at that time. They are typically defined
using the following recursion:
\begin{itemize}
  \item $\alpha_1(i) = \pi_i b_i(o_1)$, the probability of starting in
    state $i$ ($\pi_i$) times the probability of generating the first
    observation $o_1$ from state $i$.
  \item $\alpha_{t+1}(i) = b_i(o_{t+1})\sum_{j=1}^N \alpha_t(j) A_{ji}$,
    the probability of generating the observation $o_{t+1}$ from state $i$
    times the probability that we arrive in state $i$ from any other
    previous state after generating all of the other observations.
\end{itemize}

Analogously, $\beta_t(i) = P(o_{t+1}, \ldots, o_{T} \mid q_t = i, \Lambda)$
is the probability of generating the \emph{rest of the sequence} given that
we are in state $i$ at time $t$. They are also defined using a recursion:
\begin{itemize}
  \item $\beta_T(i) = 1$
  \item $\beta_t(i) = \sum_{j=1}^N \beta_{t+1}(j) A_{ij} b_j(o_{t+1})$, the
    probability of transitioning to any state $j$ and generating the
    observation $o_{t+1}$ times the probability of generating the rest of
    the sequence given that we transitioned to state $j$.
\end{itemize}

Given the $\alpha$s and the $\beta$s, we can compute $\gamma_t(i)$, the posterior
probability of being in a given state $i$ at time $t$, and $\xi_t(i,j)$,
the posterior probability of going through a transition from state $i$ to
state $j$ at time $t$ as
\begin{equation}
  \gamma_t(i) = \frac{\alpha_t(i)\beta_t(i)}{\sum_{j=1}^N
  \alpha_t(j)\beta_t(j)}
\end{equation}
and
\begin{equation}
  \xi_t(i,j) = \frac{\alpha_t(i) A_{ij} b_j(o_{t+1}) \beta_{t+1}(j)}
  {\sum_{j=1}^N \alpha_t(j)\beta_t(j)}
\end{equation}
respectively. This is the E-step of the EM algorithm.

Given $\gamma_t(i)$ and $\xi_t(i,j)$ for each sequence $\mathbf{o}_k$ in
some training data of length $M$, we can update our model parameters
$\Lambda$ as
\begin{align}
  \pi_i &= \frac{\sum_{k=1}^M \gamma^{(\mathbf{o}_k)}_1(i)}{M},\\
  A_{ij} &= \frac{\sum_{k=1}^M \sum_{t=1}^T \xi^{(\mathbf{o}_k)}_t(i,j)}
  {\sum_{k=1}^M \sum_{j=1}^N \sum_{t=1}^T \xi^{(\mathbf{o}_k)}_t(i,j)},
  \text{ and }\\
  b_i(v) &= \frac{\sum_{k=1}^M \sum_{t=1, o_{k,t} = v}^T
  \gamma^{(\mathbf{o}_k)}_t(i)}
  {\sum_{k=1}^M \sum_{t=1}^T \gamma^{(\mathbf{o}_{k,t})}_t(i)}.
\end{align}
This is the M-step of the EM algorithm.

\subsubsection{Baum-Welch for HMMs with Sequence Observations}
The major deviation of our model from the traditional HMM is that our
observations are themselves sequences. Thus, in all of the equations
presented in the previous section, we treat an observation $o_t =
\mathbf{s}_t$ at time $t$ as consisting itself of a list of actions $(a_1,
\ldots, a_M)$. Thus, instead of defining $b_i(v)$ as the probability that a
discrete observation \emph{token} $v$ is generated from state $i$, we have
\begin{equation}
  b_i(\mathbf{s}_t) = P(\mathbf{s}_t \mid \lambda^{(i)})
  = P(a_1 \mid \pi^{(i)}) \prod_{k=2}^M P(a_k \mid
  a_{k-1}, A^{(i)})
\end{equation}
where $b_i(\mathbf{s})$ now defines the probability of an action sequence
using a first-order (non-hidden) Markov model.

Fortunately, the recursions for the $\alpha$ and $\beta$ variables remain
the same, as do the definitions of $\gamma_t(i)$ and $\xi_t(i,j)$ in the
E-step. The only change is in the updating equations in the M-step,
where we replace the update for $b_i(v)$ by a pair of updates: one for
$\pi^{(i)}_a$ one for $A^{(i)}_{ab}$. We denote the $m$-th action in the
$t$-th sequence for the $k$-th student as $o_{k,t,m}$. The two updates can
be written as
\begin{align}
  \pi^{(i)}_{a}
  &= \frac{\sum_{k=1}^M \sum_{t=1,a_1 = a}^T \gamma^{(\mathbf{o}_k)}_t(i)}
  {\sum_{k=1}^M \sum_{t=1}^T \gamma^{(\mathbf{o}_k)}_t(i)}, \text{ and }\\
  A^{(i)}_{ab}
  &= \frac{\sum_{k=1}^M \sum_{t=1}^T
  \sum_{m=2,a_{m-1}=a \land a_m = b}^{|\mathbf{o}_{k,t}|}
  \gamma^{(\mathbf{o}_k)}_t(i)}
  {\sum_{k=1}^M \sum_{t=1}^T \sum_{m=2}^{|\mathbf{o}_{k,t}|}
  \gamma^{(\mathbf{o}_k)}_t(i)}
\end{align}
where $a_m = o_{t,k,m}$ (for notational conciseness). Our modified EM
algorithm for HMMs with sequence observations is provided as part of the
\meta/ toolkit~\citemeta{}.
