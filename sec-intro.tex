\section{Introduction}
The proliferation of massive open online courses (MOOCs) has resulted in a
profound impact on education. As more and more learners turn to MOOCs to
educate themselves on various topics, more and more behavioral data is
being collected as part of the system on which the MOOC is offered. The
data present in these logs has the power to aid us in understanding the
behavior of students who take our MOOCs, which is mostly undetectable for
instructors of these MOOCs today due to its vast scale. As a result, the
rich data available through these MOOC logs is highly underutilized today.

What stands in the way? Clearly, intelligent systems to create concise and
digestible summaries of the massive amount of interaction data collected
are needed in order to empower the instructors of these courses. If we can
understand how users are interacting with our MOOCs, we are much more
likely to be able to make changes to these courses that positively impact
learners. While we can easily observe the changes in behavior of students
in real classrooms, MOOCs present a challenge due to their hands-off
nature and sometimes irregular schedule due to being a full-time worker.   
We view this paper as attempting to bridge this gap. Specifically,
in this paper we propose unsupervised learning methods for automatically
discovering and characterizing student learning behavior patterns or profiles from large
collections of click logs associated with MOOCs. 

Our work is motivated by the following observations:
\begin{enumerate}
  \item Student behavior is complicated and cannot necessarily be captured
      sufficiently by rule-based methods such as those explored by
      \citet{Kizilcec:2013:LAK} and \citet{Davis:2016:EDM}. We instead
      propose to treat student behavior patterns as being characterized
      (represented) via a sequence of \emph{latent states}. This allows us
      to automatically capture patterns that we might not have been able to
      articulate clearly a priori via a series of rules, and also allows us
      to model the inherent uncertainty in assigning a student's behavior
      to a particular pattern or group.
  \item Student behavior can vary over time. Previous models that treat students
      as exhibiting only one behavioral pattern over
      time~\cite{Faucon:2016:EDM} miss out on the opportunity to understand
      student behavior dynamics in a course. We propose a latent space
      model with {\em latent state transitions} to flexibly model the
      dynamics.
  \item Analysis of student behavior can and should be performed at varying
      levels of granularity. This requires us to aggregate data over time
      with \emph{different levels of resolution}; existing models tend to come
      with a particular assumption about the resolution of time they
      consider~\cite{Faucon:2016:EDM, Kizilcec:2013:LAK, Shih:2010:EDM}. We
      propose a more flexible model to accommodate different levels of
      resolution.
\end{enumerate}

Thus, what we propose is a \emph{latent variable approach} to mining student behavior
patterns that is \emph{probabilistic} for inference and 
\emph{flexible to model state changes over different time resolutions}. 
More specifically, we propose a novel two-layer hidden Markov model (TL-HMM) to
 discover latent student behavior patterns via
unsupervised learning on large collections of student behavior observation
sequences. Evaluation results on a MOOC data set on Coursera demonstrate that TL-HMM can effectively 
discover a variety of interesting interpretable student behavior patterns at different levels of resolution, many of which are beyond what existing approaches can discover. TL-HMM further enables easy use of the discovered patterns for discriminative analysis such as prediction of learning outcome. 
Since our proposed methods are unsupervised, they can potentially be applied to any MOOC data without requiring any manual work to facilitate understanding of student behaviors and their variations, opening up many possibilities for developing intelligent tutoring systems that can adaptive to student behavior. 


