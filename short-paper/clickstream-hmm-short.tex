\documentclass{sigchi-ext}

% Please be sure that you have the dependencies (i.e., additional
% LaTeX packages) to compile this example.
\usepackage[T1]{fontenc}
\usepackage{textcomp}
\usepackage[scaled=.92]{helvet} % for proper fonts
\usepackage{graphicx} % for EPS use the graphics package instead
\usepackage{balance}  % for useful for balancing the last columns
\usepackage{booktabs} % for pretty table rules
\usepackage{ccicons}  % for Creative Commons citation icons
\usepackage{ragged2e} % for tighter hyphenation

% Some optional stuff you might like/need.
% \usepackage{marginnote}
% \usepackage[shortlabels]{enumitem}
% \usepackage{paralist}
\usepackage[utf8]{inputenc} % for a UTF8 editor only
\usepackage{amsmath}
\usepackage{float}
\usepackage[numbers]{natbib}

\usepackage{subcaption}
\usepackage{pdflscape}
\usepackage{afterpage}

\renewcommand\cite{\citep}
\def\meta/{\textsc{MeTA}}
\newcommand{\textretrieval}{textretrieval-001}
\newcommand{\sustain}{sustain-001}
\newcommand{\UIUC}{UIUC}

%% EXAMPLE BEGIN -- HOW TO OVERRIDE THE DEFAULT COPYRIGHT STRIP --
% \copyrightinfo{Permission to make digital or hard copies of all or
% part of this work for personal or classroom use is granted without
% fee provided that copies are not made or distributed for profit or
% commercial advantage and that copies bear this notice and the full
% citation on the first page. Copyrights for components of this work
% owned by others than ACM must be honored. Abstracting with credit is
% permitted. To copy otherwise, or republish, to post on servers or to
% redistribute to lists, requires prior specific permission and/or a
% fee. Request permissions from permissions@acm.org.\\
% {\emph{CHI'14}}, April 26--May 1, 2014, Toronto, Canada. \\
% Copyright \copyright~2014 ACM ISBN/14/04...\$15.00. \\
% DOI string from ACM form confirmation}
%% EXAMPLE END

\copyrightinfo{Permission to make digital or hard copies of all or part of
  this work for personal or classroom use is granted without fee provided
  that copies are not made or distributed for profit or commercial
  advantage and that copies bear this notice and the full citation on the
  first page. Copyrights for components of this work owned by others than
  the author(s) must be honored. Abstracting with credit is permitted. To
  copy otherwise, or republish, to post on servers or to redistribute to
  lists, requires prior specific permission and/or a fee. Request
  permissions from Permissions@acm.org.\\
  \emph{L@S 2017}, April 20 - 21, 2017, Cambridge, MA, USA.\\
  Copyright is held by the owner/author(s). Publication rights licensed to
  ACM.\\
  ACM 978-1-4503-4450-0/17/04 \$15.00\\
  DOI: \url{http://dx.doi.org/10.1145/3051457.3053986}}

% Paper metadata (use plain text, for PDF inclusion and later
% re-using, if desired).  Use \emtpyauthor when submitting for review
% so you remain anonymous.
\def\plaintitle{Modeling MOOC Student Behavior with Two-Layer Hidden Markov
Models}
\def\plainauthor{Chase Geigle, ChengXiang Zhai}
\def\emptyauthor{}
\def\plainkeywords{MOOC log analysis; studetn behavior modeling; Markov models; hidden Markov models}
\def\plaingeneralterms{\plainkeywords}

\title{\plaintitle}

\numberofauthors{2}
% Notice how author names are alternately typesetted to appear ordered
% in 2-column format; i.e., the first 4 autors on the first column and
% the other 4 auhors on the second column. Actually, it's up to you to
% strictly adhere to this author notation.
\author{%
  \alignauthor{%
    \textbf{Chase Geigle}\\
    \affaddr{University of Illinois at Urbana-Champaign} \\
    \affaddr{201 N Goodwin Ave}\\
    \affaddr{Urbana, IL 61801} \\
    \email{geigle1@illinois.edu} } \vfil \alignauthor{%
    \textbf{ChengXiang Zhai}\\
    \affaddr{University of Illinois at Urbana-Champaign} \\
    \affaddr{201 N Goodwin Ave}\\
    \affaddr{Urbana, IL 61801} \\
    \email{czhai@illinois.edu} } }

% Make sure hyperref comes last of your loaded packages, to give it a
% fighting chance of not being over-written, since its job is to
% redefine many LaTeX commands.
\definecolor{linkColor}{RGB}{6,125,233}
\hypersetup{%
  pdftitle={\plaintitle},
%  pdfauthor={\plainauthor},
  pdfauthor={\emptyauthor},
  pdfkeywords={\plainkeywords},
  bookmarksnumbered,
  pdfstartview={FitH},
  colorlinks,
  citecolor=black,
  filecolor=black,
  linkcolor=black,
  urlcolor=linkColor,
  breaklinks=true,
}

% \reversemarginpar%

\newcommand{\citemeta}{\cite{Massung:2016:ACL}}

\begin{document}

\maketitle

% Uncomment to disable hyphenation (not recommended)
% https://twitter.com/anjirokhan/status/546046683331973120
\RaggedRight{}

% Do not change the page size or page settings.
\begin{abstract}
  Massive open online courses (MOOCs) provide educators with an abundance
  of data describing how students interact with the platform, but this data
  is highly underutilized today. This is in part due to the lack of
  sophisticated tools to provide interpretable and actionable summaries of
  huge amounts of MOOC activity present in log data. In this paper, we
  propose a method for automatically discovering student behavior patterns
  by leveraging the click log data that can be obtained from the MOOC
  platform itself in a completely unsupervised manner.
\end{abstract}

\keywords{\plainkeywords}

\begin{CCSXML}
  <ccs2012>
    <concept>
      <concept_id>10002950.10003648.10003670.10003683</concept_id>
      <concept_desc>Mathematics of computing~Kalman filters and hidden Markov models</concept_desc>
      <concept_significance>500</concept_significance>
    </concept>
    <concept>
      <concept_id>10010147.10010257.10010258.10010260.10010267</concept_id>
      <concept_desc>Computing methodologies~Mixture modeling</concept_desc>
      <concept_significance>500</concept_significance>
    </concept>
  </ccs2012>
\end{CCSXML}

\ccsdesc[500]{Mathematics of computing~Kalman filters and hidden Markov models}
\ccsdesc[500]{Computing methodologies~Mixture modeling}

\printccsdesc{}

\section{Introduction}
The proliferation of massive open online courses (MOOCs) has resulted in a
profound impact on education. As more and more learners turn to MOOCs to
educate themselves on various topics, more and more behavioral data is
being collected as part of the system on which the MOOC is offered. The
data present in these logs has the power to aid us in understanding the
behavior of students who take our MOOCs, which is mostly undetectable for
instructors of these MOOCs today due to its vast scale. As a result, the
rich data available through these MOOC logs is highly underutilized today.

What stands in the way? Clearly, intelligent systems to create concise and
digestible summaries of the massive amount of interaction data collected
are needed in order to empower the instructors of these courses. If we can
understand how users are interacting with our MOOCs, we are much more
likely to be able to make changes to these courses that positively impact
learners. While we can easily observe the changes in behavior of students
in real classrooms, MOOCs present a challenge due to their hands-off
nature and sometimes irregular schedule due to being a full-time worker.   
We view this paper as attempting to bridge this gap. Specifically,
in this paper we propose unsupervised learning methods for automatically
discovering and characterizing student learning behavior patterns or profiles from large
collections of click logs associated with MOOCs. 

Our work is motivated by the following observations:
\begin{enumerate}[itemsep=2pt,topsep=2pt]
  \item Student behavior is complicated and cannot necessarily be captured
      sufficiently by rule-based methods such as those explored by
      \citet{Kizilcec:2013:LAK}. We instead propose to treat
      student behavior patterns as being characterized (represented) via a sequence of
      \emph{latent states}. This allows us to automatically capture
      patterns that we might not have been able to articulate clearly a
      priori via a series of rules, and also allows us to model the
      inherent uncertainty in assigning a student's behavior to a
      particular pattern or group. 
  \item Student behavior can vary over time. Previous models that treat students
      as exhibiting only one behavioral pattern over
      time~\cite{Faucon:2016:EDM} miss out on the opportunity to understand
      student behavior dynamics in a course. We propose a latent space model to flexibly model the dynamics. 
  \item Analysis of student behavior can and should be performed at varying
      levels of granularity. This requires us to aggregate data over time
      with \emph{different levels of resolution}; existing models tend to come
      with a particular assumption about the resolution of time they
      consider~\cite{Faucon:2016:EDM, Kizilcec:2013:LAK, Shih:2010:EDM}. We propose a more flexible model to accommodate different levels of resolution. 
\end{enumerate}

Thus, what we propose is a \emph{latent variable approach} to mining student behavior
patterns that is \emph{probabilistic} for inference and 
\emph{flexible to model state changes over different time resolutions}. 
More specifically, we propose a novel two-layer hidden Markov model (TL-HMM) to
 discover latent student behavior patterns via
unsupervised learning on large collections of student behavior observation
sequences. Evaluation results on a MOOC data set on Coursera demonstrate that TL-HMM can effectively 
discover a variety of interesting interpretable student behavior patterns at different levels of resolution, many of which are beyond what existing approaches can discover. TL-HMM further enables easy use of the discovered patterns for discriminative analysis such as prediction of learning outcome. 
Since our proposed methods are unsupervised, they can potentially be applied to any MOOC data without requiring any manual work to facilitate understanding of student behaviors and their variations, opening up many possibilities for developing intelligent tutoring systems that can adaptive to student behavior. 



\section{A Two-Layer HMM for MOOC Log Analysis}

Our general idea is to use a probabilistic generative model to model 
the student activities as recorded in a MOOC log, which means we will 
assume that all the observed student activities are samples drawn (i.e., ``generated'') from 
a parameterized probabilistic model. We can then estimate the 
parameter values of the probabilistic model by fitting the model to a specific
MOOC log data set. The estimated parameter values could then be treated as 
the latent ``knowledge'' discovered from the data. Because such a generative model 
attempts to fit {\em all} the data, it enables us to discover interesting patterns
that can explain the {\em overall} behavior of a student or the {\em common} behavior patterns shared by many students. 

An HMM  is a specific probabilistic generative model with a ``built-in'' state transition system
that would control the data to be generated by the model, thus it is especially 
suitable for modeling sequence data~\cite{Rabiner:1990:RSR}. At any moment, the HMM would be in 
one of $k$ states $U=\{u_1, ...,u_k\}$, and at the next moment, the HMM would move to 
another state stochastically according to a transition matrix that specifies the probability of
going to state $u_i$ when the HMM is currently in state $u_j$, i.e., $p(u_i|u_j)$. 
When the HMM is in state $u$, the HMM can generate an observable data point $x$ 
according to an output probabilistic model $p(x|u)$. Thus if we ``run'' an HMM for 
$N$ time points denoted by $t=1, ..., N$, the HMM could ``generate'' a sequence of 
observations $x_1 ... x_N$, where each $x_i$ is an output symbol by 
going through a sequence of {\em hidden states} $w_1 ... w_N$ where $w_i \in U$ is a state. 
The association of such a latent sequence of state transitions with the observed symbols makes
it possible to use HMM to ``decode'' the latent behavior of students behind the surface behavior we directly observe in the log data, allowing for understanding student behavior more deeply than a model with no latent state variables. 

In many ways, the generation process behind an HMM is meant to simulate the
actual behavior of a student. We assume that students transition through different ``task states'' (or ``behavior states'')  in the process of study. 
One such task state may be to learn about a topic by mostly watching
lecture videos, another task state may be to work on quizzes, and yet
another may be to participate in forum discussions. While in each of these
different states, the student would tend to exhibit different surface
``micro'' behaviors. For example, in the lecture study state, the student
would tend to have many video-watching related behaviors and occasionally
forum activities, while in the quiz-taking state (in order to pass each
module), the student would tend to show many quiz-related ``micro''
activities as well as asking questions or checking discussions on the
forum. Note that due to the complexity of the student behavior, it is very
difficult to accurately {\em prescribe} the specific surface ``micro''
behavior patterns for each state in advance, especially without  prior knowledge about the students. For example, forum activities are likely interleaved with other activities   in every task state and the interleaving pattern can be somewhat irregular with potentially many variations. 
The major motivations for using an HMM are that 
\begin{enumerate*}[label=(\arabic*)]
    \item it uses a probabilistic model (i.e., the output probability distribution $p(x|u)$ conditioned on each state) to directly capture the inevitable uncertainty in the association of surface ``micro'' activities with their corresponding latent task/behavior state, which is often our main target to discover and characterize, and 
    \item it does not make any assumption about which latent task/behavior
      state must be associated with which observed activities or how a
      student would move from one state to another, but instead allows our
      data to ``tell'' us what kind of associations are most likely,
      what kind of transitions are most probable, and which states tend to be more long-lasting for any particular set of students. 
\end{enumerate*}

However, if we use an ordinary HMM to analyze our data, we would treat each
observed ``micro'' activity (e.g., video watching, or forum post reading)
as an output symbol, and thus the output distribution $p(x|u)$ for each
discovered latent state would be a simple distribution over all kinds of
observable micro activities recorded in our log data (e.g., 50\% lecture
watching, 8\% quiz taking, 7\% quiz submission, 2\% course wiki reading,
...). While such a distribution is meaningful and can already help us
interpret the corresponding latent state, it only gives us a rather
superficial characterization of student behavior.

Ideally, we want $p(x|u)$ to characterize the directly observable ``micro''
behavior in more detail to further capture the relations and dependencies
of these micro activities. To this end, we would treat an {\em entire
sequence} of micro activities (e.g., one session of activities) as an
observed ``symbol'' from a latent state, and further model the generation
of such a sequence with another Markov model where we treat each micro
activity as an {\em observable} state, and model the transitions between
these activity states in very much the same way as the state transitions in
HMM.

Adding this second layer would allow us to characterize a latent task
state in much more detail, as it would reveal not only what activities are
most common to a task state, but also the transition patterns between these
``micro'' activities (e.g., it can reveal frequent back-and-forth
transitions between quiz-taking and quiz-submission, which would suggest a
concentrated period of taking quizzes). Combining this ``surface'' Markov
model over the ``micro'' actions with the ``deep'' hidden Markov model over
the latent task states gives us a fairly general and powerful two-layer HMM
(TL-HMM) that can simultaneously learn ``deeply'' the latent task/behavior
states and their transitions as well as the corresponding ``micro''
activity transition patterns associated with each latent state to
facilitate interpretation and analysis of the discovered latent state
patterns. Our implementation of the learning algorithm for TL-HMMs is
included as part of the \meta/ toolkit~\citemeta{}.

\newcommand{\textretrieval}{\preprintonly{textretrieval-001}%
    \finalonly{textretrieval-001}%
    \submissiononly{moocname1-00X}}

\newcommand{\sustain}{\preprintonly{sustain-001}%
    \finalonly{sustain-001}%
    \submissiononly{moocname2-00X}}

\newcommand{\UIUC}{\preprintonly{UIUC}%
    \finalonly{UIUC}%
    \submissiononly{(redacted University)}}

\section{Results}
To qualitatively evaluate our model, we look at the latent state
representations we learn by fitting the model to the log data associated
with two different Coursera MOOCs: \textretrieval and \sustain.

\section{Conclusions and Future Work}
% summarize the main contributions of the paper; summarize the experimental
% results; argue again for our novelty; discuss future directions
As a tool to help instructors and education researchers better understand
the behavior of MOOC students, we proposed a two-layer hidden Markov model
to automatically extract student activity patterns in the form of behavior
state-transition graphs from large amounts of MOOC log data. This model is
different from existing methods in that it treats behavior patterns as a
sequence of \emph{latent states}, rather than assigning these states in a
rule-based manner. It captures the variable behaviors of students over time
and allows analysis at different levels of granularity.

We showed that such a model does, in fact, capture meaningful behavior
patterns and produces descriptions of these behavior patterns that are easy
to interpret. We argued that it is important to capture student behavior
patterns with more sophisticated models than simple discrete distributions
over actions in order to capture information present in bigrams of actions
(or above). By varying the number of latent
states inferred, we showed that the model is flexible and can capture
patterns of differing levels of specificity in this way.  Finally, we
investigated whether we can detect differences in student behavior patterns
as they correlate with course performance. Specifically, we demonstrated
that high-performing students produce substantially different HMM
transition diagrams that tend to show longer concentration span in quiz-taking
and more active  forum participation as compared with the average students.
These results show the great potential of the proposed model for serving as a tool
to help humans discover knowledge about student behaviors.  

%\subsection{Future Work}

There are a number of interesting directions to further extend our work in the future. 
First, the proposed model is 
completely general and can thus be easily applied to analyze the log of many
other courses to enable deep understanding of student behaviors as well as
the correlations of such behaviors and other variables such as grades. To realize these benefits, 
it would be useful 
to develop a MOOC log analysis system based on the proposed model so as to facilitate education research and help instructors improve course
design.

Second, our model can empower many new comparative analyses. For example, we could
now look at how behavior patterns change between different offerings of
the same MOOC to understand how changes in course structure or materials
influence student behavior. Individual students can now also be compared
against each other or against groups. For example, by decoding the latent
state sequences for each student, we can measure how ``surprising'' their
latent state transition sequence is relative to the average we would expect
according to the model, or to the average ``perfect'' student, etc. We can
now investigate how certain behavioral patterns correlate with properties
of a student (e.g., demographics, prior aptitude, etc.). After decoding the
students' latent state sequences, we could also correlate course-wide
drifts in these latent states with events in a course. For example, we
might be able to automatically discover difficult or confusing parts of a
course by noticing spikes in the distribution of students over latent
states over time.

Third, there is more recent work on better learning algorithms for mixtures of
Markov models~\citep{Gupta:2016:NIPS}. It would be worth exploring whether
the advances proposed in this and similar work can be applied to our model
to address some of the concerns surrounding our use of the EM algorithm for
our parameter estimation.

Finally, the proposed model can be extended in several ways. For example, 
although our model does not explicitly model drop-out like
\citet{Kizilcec:2013:LAK}, doing so is an obvious extension. Our model
would be able to provide predictions of when a student is ``at risk'' for
dropping out under such an extension.
Also, currently, the model learns a transition matrix over the latent states that
is \emph{shared} across all students. It would be interesting to instead
learn a different latent state transition matrix for each individual
student, but keep the second-level Markov models shared. This would provide
the model with more flexibility which may be desirable itself, but would
also naturally result in a description of a student (via his or her HMM
transitions) that could be incorporated into existing supervised learning
techniques that try to predict student outcomes for understanding
which of the latent behavior patterns discovered by 2L-HMM are most predictive
of the performance of student learning. One could also relax this somewhat
and posit that \emph{groups} of students transition between the lower layer
patterns identified by our method in distinct ways; this way, we can do a soft
clustering of students into $K_2$ groups based on the similarity of their
transitioning behavior between the higher-level behaviors we can identify.




\section{Acknowledgments}

This material is based upon work supported by the NSF GRFP under Grant
Number DGE-1144245.

\balance{}

\bibliographystyle{acm-sigchi-modern}
\bibliography{bib}

\end{document}
