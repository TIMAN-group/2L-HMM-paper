%% SIGCHI Proceedings Format (modern) Sample Document
%% This sample file demonstrates the use of the `sigchi-modern' class file.
%%
%% The content of this document draws heavily on the `HCI Archive Format'
%% template supplied by SIGCHI, but has been modified to demonstrate and
%% document the LaTeX class.

% The parameter given to the `\documentclass' command can be one of:
%   `preprint', `submission', or `final'.
% These control the presence of certain features for each stage of the authoring
% process. For example, `submission' does not output the author block, and
% `final' disables page numbers.
%
% The \preprintonly{}, \submissiononly{}, and \finalonly{} commands allow you
% to wrap text that will only be output under the corresponding option.
\documentclass[preprint]{sigchi-modern}

% These packages aren't required, but do provide some helpful features that are
% used in this sample.
\usepackage{graphicx}  % Graphics for figures
\usepackage{booktabs}  % Nice formatting for tables
\usepackage{balance}   % Attempts to balance the columns on the last page

% A nice way to help avoid overful lines handing out into the margins is the
% `microtype' package, which lets certain characters extrude a *little* bit.
\usepackage{microtype}
% If the situation becomes dire, use:
%\sloppy

% You may use any bibliography package as long as the output matches that of
% the prescribed format. `natbib' is a fairly nice default, and the provided
% BibTeX style file has been tested with it.
\usepackage[numbers]{natbib}
\bibliographystyle{acm-sigchi-modern}

% Define these tokens as appropriate for the conference to generate the
% necessary copyright notice for `final` mode.
\confname{CHI'14}
\confdate{April 26--May 1}
\confyear{2014}
\conflocation{Toronto, Canada}
\procissn{XXX-X-XXXX-XXXX-X/XX/XX}
\doi{10.1000/182}
% \copylicense be set to one of: \acmcopyright, \authorlicense, or \openlicense
\copylicense{\authorlicense}

% End of preamble. Here comes the document.
\begin{document}

\title{Modeling MOOC Student Behavior With Two-Layer Hidden Markov Models}

% Author information can be set automatically if you provide a series of
% author names and affiliations:
%   \author[1]{Author One}
%   \author[2]{Author Two}
%   \author[1,2]{Author Three}
%   \affiliation[1]{Organisation One\\
%                   Some Country}
%   \affiliation[2]{Organisation Two\\
%                   Some Country}
%   \authorextra{\mailto{one@example.com}, \mailto{two@example.org},
%                \mailto{three@example.net}}
% Or, manually:
%   \author{
%     \authorname{Author One}\\
%     \authoraffil{
%       Organisation One\\
%       Some Country\\
%       \mailto{one@example.com}}}
%   \author{
%     \authorname{Author Two}\\
%     \authoraffil{
%       Organisation Two\\
%       Some Country\\
%       \mailto{two@example.org}}}
% Refer to the body text of the document for more details.
\author[1]{Chase Geigle}
\author[1]{Hao Zheng}
\author[1]{ChengXiang Zhai}

% Set \authorwidth with the longest author name if you don't like the default
% spacing:
\settowidth{\authorwidth}{ChengXiang Zhai1}
\settowidth{\affilwidth}{University of Illinois at Urbana-ChampaignAAA}

\affiliation[1]{Department of Computer Science\\
    University of Illinois at Urbana-Champaign\\
    Urbana, Illinois, USA}

\authorpostscript{\email{\{%
  \href{mailto:geigle1@illinois.edu}{geigle1},
  \href{mailto:hzheng19@illinois.edu}{hzheng19},
  \href{mailto:czhai@illinois.edu}{czhai}\}@illinois.edu}}

% If you format the author blocks manually, use \authorlist with a
% comma-delimited list of the author names.
% This is written into the PDF metadata for preprint and final modes.
%\authorlist{Author One, Author Two, Author Three, Author Four}

% Produces the title, author, and optional banner block.
\maketitle

\begin{abstract}
  Massive open online courses (MOOCs) provide educators with an abundance
  of data describing how students interact with the platform, but this data
  is highly underutilized today. This is in part due to the lack of
  sophisticated tools to provide interpretable and actionable summaries of
  huge amounts of MOOC activity present in log data. In this paper, we
  propose a method for automatically discovering student behavior patterns
  by leveraging the click log data that can be obtained from the MOOC
  platform itself in a completely unsupervised manner.
\end{abstract}

% A semicolon-separated list of keywords to describe the paper.
\keywords{MOOC log analysis; student behavior modeling; Markov models;
hidden Markov models}

% ACM classification block.
% \category{}{}{}[] has three mandatory arguments and one optional argument; it
% can be used multiple times.
%\classification{%
%    \category{}{}{}
%}

\section{Introduction}
The proliferation of massive open online courses (MOOCs) has resulted in a
profound impact on education. As more and more learners turn to MOOCs to
educate themselves on various topics, more and more behavioral data is
being collected as part of the system on which the MOOC is offered. The
data present in these logs has the power to aid us in understanding the
behavior of students who take our MOOCs, which is mostly undetectable for
instructors of these MOOCs today due to its vast scale. As a result, the
rich data available through these MOOC logs is highly underutilized today.

What stands in the way? Clearly, intelligent systems to create concise and
digestible summaries of the massive amount of interaction data collected
are needed in order to empower the instructors of these courses. If we can
understand how users are interacting with our MOOCs, we are much more
likely to be able to make changes to these courses that positively impact
learners. While we can easily observe the changes in behavior of students
in real classrooms, MOOCs present a challenge due to their hands-off
nature and sometimes irregular schedule due to being a full-time worker.   
We view this paper as attempting to bridge this gap. Specifically,
in this paper we propose unsupervised learning methods for automatically
discovering and characterizing student learning behavior patterns or profiles from large
collections of click logs associated with MOOCs. 

Our work is motivated by the following observations:
\begin{enumerate}[itemsep=2pt,topsep=2pt]
  \item Student behavior is complicated and cannot necessarily be captured
      sufficiently by rule-based methods such as those explored by
      \citet{Kizilcec:2013:LAK}. We instead propose to treat
      student behavior patterns as being characterized (represented) via a sequence of
      \emph{latent states}. This allows us to automatically capture
      patterns that we might not have been able to articulate clearly a
      priori via a series of rules, and also allows us to model the
      inherent uncertainty in assigning a student's behavior to a
      particular pattern or group. 
  \item Student behavior can vary over time. Previous models that treat students
      as exhibiting only one behavioral pattern over
      time~\cite{Faucon:2016:EDM} miss out on the opportunity to understand
      student behavior dynamics in a course. We propose a latent space model to flexibly model the dynamics. 
  \item Analysis of student behavior can and should be performed at varying
      levels of granularity. This requires us to aggregate data over time
      with \emph{different levels of resolution}; existing models tend to come
      with a particular assumption about the resolution of time they
      consider~\cite{Faucon:2016:EDM, Kizilcec:2013:LAK, Shih:2010:EDM}. We propose a more flexible model to accommodate different levels of resolution. 
\end{enumerate}

Thus, what we propose is a \emph{latent variable approach} to mining student behavior
patterns that is \emph{probabilistic} for inference and 
\emph{flexible to model state changes over different time resolutions}. 
More specifically, we propose a novel two-layer hidden Markov model (TL-HMM) to
 discover latent student behavior patterns via
unsupervised learning on large collections of student behavior observation
sequences. Evaluation results on a MOOC data set on Coursera demonstrate that TL-HMM can effectively 
discover a variety of interesting interpretable student behavior patterns at different levels of resolution, many of which are beyond what existing approaches can discover. TL-HMM further enables easy use of the discovered patterns for discriminative analysis such as prediction of learning outcome. 
Since our proposed methods are unsupervised, they can potentially be applied to any MOOC data without requiring any manual work to facilitate understanding of student behaviors and their variations, opening up many possibilities for developing intelligent tutoring systems that can adaptive to student behavior. 




\section{Related Work}

Our model is based heavily upon the prior art of Hidden Markov
Models~\cite{Rabiner:1990:RSR} for sequence labeling tasks. These types of
models are widely applicable and have been used for tasks such as speech
recognition~\cite{Huang:1990:HMM}, part-of-speech
tagging~\cite{Jurafsky:2009:SLP}, and econometrics~\cite{Hamilton:1990:JoE}
and are a member of the more general family of probabilistic graphical
models~\cite{Koller:2009:PGM}.  A major challenge in applying HMMs and
other graphical models successfully to solve a problem is to design an
appropriate architecture of the model, which always varies according to
specific applications.

A multi-layered approach to HMM modeling of sequence data has been
performed before in other domains. \citet{Zhang:2004:CVPR}, for example,
explored a two-layer HMM framework for modeling actions in meetings, but
their definition of ``two-layer'' differs from ours. In their formulation,
the ``lower-layer'' level is used to label audio-video action sequences
into basic events, and the ``upper-layer'' is used to label the output of
the lower-layer to discover higher-level office behavior abstractions. Our
formulation differs in that we do not feed the labeled sequence of the
lower level into the input of the higher level. Instead, our lower level is
actually treated using a non-hidden Markov model, and the higher level is
modeling transitions between the $K$ different non-hidden Markov models we
consider.

Our formulation is more closely related to the Hierarchical Hidden Markov
Model (HHMM) detailed in \citet{Fine:1998:ML}. Here, the ``layers'' are
modeled by having the hidden Markov model have two kinds of transitions.
Horizontal transitions move between states within a layer, where vertical
transitions move between different layers. At the bottom layer lie the
``production'' states, which output symbols according to some probability
distribution. Our specific model in this case can be modeled as a HHMM
where the horizontal transitions between nodes at the highest layer
(including self-loops) \emph{must} be immediately followed by a vertical
transition to the lower layer. The output probability distributions over
symbols in the lower layer ``production'' level are forced to emit only one
kind of symbol, and vertical transitions are only allowed into the original
higher-layer state that transitioned down into the lower-layer.

Mixtures of hidden Markov models are also conceptually similar to our
formulation. \citet{Song:2009:NDSS} explored using a mixture of hidden
Markov models in the context of anomaly detection in the security domain.
\citet{Ypma:2002:Springer} use mixtures of HMMs to categorize web pages and
cluster users by investigating web log data, which is quite similar to the
clickstream log data we obtain from MOOCs. The major difference between
our approach and a standard mixture of HMMs is that we also model the
\emph{transition behavior} between the Markov models that make up our
model's lowest layer, where a standard mixture of HMMs would ignore the
potential dependence of the previous sequence's latent state on the next
sequence's latent state.

HMMs or similar ideas have been previously applied to model education
data~\cite{Shih:2010:EDM,Kizilcec:2013:LAK,Davis:2016:EDM}, but the
previous models are not well tuned toward the student behavior task and
thus cannot adequately address all the aspects of complexity of student
learning behavior.  A main technical contribution of this paper is to
propose a more general HMM that can better adapt to the variations of
student behavior via its variable resolution and nested HMM structure, and
thus enable discovery of more sophisticated behavior patterns and provide
more detailed characterization of student behavior than the previous
models.

For example, \citet{Kizilcec:2013:LAK} assigned students to states
following a rule-based approach based upon when the student submitted the
assignment for a particular week in the course. They investigated how
students transitioned between these states as the course progressed, and
used the sequence of states a student exhibited as a representation for
performing $k$-means clustering of students into related groups. This
differs from our method substantially: we assign students to states using a
probabilistic framework to account for uncertainty in this state
assignment, and jointly learn \emph{representations} for these states,
which are treated as being \emph{latent} as opposed to pre-defined using
some rule (or set of rules).  Furthermore, our model provides more
flexibility in how the time segments are defined, allowing for both finer
(for example, day-by-day) or coarser (for example, month-by-month)
granularity. \citet{Shih:2010:EDM} investigated the use of HMM-based
clustering techniques for automatic discovery of student learning
strategies when solving a particular problem.  This is similar to our
approach in that the description of behavior profiles is a Markov model,
but cannot further characterize each latent state with another informative
HMM. Thus their work can be regarded to modeling ``micro'' behavior,
whereas our model can model both ``micro'' and ``macro'' behavior.

\citet{Davis:2016:EDM} investigate frequent student behavior pattern chains
with a set of actions that is defined similarly to ours. However, their
method for finding the common behavioral patterns involves a manual
clustering step to identify behavioral ``motifs,'' which is then followed
by an automatic (rule-based) assignment of all sequences to these motifs.
Our method, by contrast, attempts to do this automatically: the latent
state representations obtained by our model attempt to capture similar
meanings to their behavior motifs in a completely automated fashion. They
also automatically generate and investigate Markov models for different
MOOCs, but do so by considering \emph{all} student action sequences as
belonging to a \emph{single Markov model}. In our approach, we allow each
student behavior sequence to belong to one of $K$ different Markov models
(and further model the transition probabilities between these latent state
Markov models between each sequence a student generates). Thus, their
Markov models presented are a special case of our model when $K = 1$.

\citet{Faucon:2016:EDM} proposed a semi-Markov model for simulating MOOC
students. They produce behavior profiles that characterize groups of
students in the form of semi-Markov models like our proposed model does,
but they assume that a student can belong to only one behavior profile
across the entire course rather than allowing this profile to change over
time. Because we do not have this restriction, our model is also able to
learn the transition probabilities between the different behavior profiles
we discover.

There are a few additional related studies worth mentioning.
Bayesian Knowledge Tracing~\cite{Corbett:1994:UMUAI} in its basic form uses
a hidden Markov model to model the probability that a learner knows a
certain skill at a given time. Modifications to this algorithm include
contextual estimation of the ``slip'' and ``guess'' probabilities of the
model~\cite{Baker:2008:ITS} and most recently a re-framing as a neural
network problem~\cite{Piech:2015:NIPS}.



\section{A Two-Layer HMM for MOOC Log Analysis}

Our general idea is to use a probabilistic generative model to model 
the student activities as recorded in a MOOC log, which means we will 
assume that all the observed student activities are samples drawn (i.e., ``generated'') from 
a parameterized probabilistic model. We can then estimate the 
parameter values of the probabilistic model by fitting the model to a specific
MOOC log data set. The estimated parameter values could then be treated as 
the latent ``knowledge'' discovered from the data. Because such a generative model 
attempts to fit {\em all} the data, it enables us to discover interesting patterns
that can explain the {\em overall} behavior of a student or the {\em common} behavior patterns shared by many students. 

An HMM  is a specific probabilistic generative model with a ``built-in'' state transition system
that would control the data to be generated by the model, thus it is especially 
suitable for modeling sequence data~\cite{Rabiner:1990:RSR}. At any moment, the HMM would be in 
one of $k$ states $U=\{u_1, ...,u_k\}$, and at the next moment, the HMM would move to 
another state stochastically according to a transition matrix that specifies the probability of
going to state $u_i$ when the HMM is currently in state $u_j$, i.e., $p(u_i|u_j)$. 
When the HMM is in state $u$, the HMM can generate an observable data point $x$ 
according to an output probabilistic model $p(x|u)$. Thus if we ``run'' an HMM for 
$N$ time points denoted by $t=1, ..., N$, the HMM could ``generate'' a sequence of 
observations $x_1 ... x_N$, where each $x_i$ is an output symbol by 
going through a sequence of {\em hidden states} $w_1 ... w_N$ where $w_i \in U$ is a state. 
The association of such a latent sequence of state transitions with the observed symbols makes
it possible to use HMM to ``decode'' the latent behavior of students behind the surface behavior we directly observe in the log data, allowing for understanding student behavior more deeply than a model with no latent state variables. 

In many ways, the generation process behind an HMM is meant to simulate the
actual behavior of a student. We assume that students transition through different ``task states'' (or ``behavior states'')  in the process of study. 
One such task state may be to learn about a topic by mostly watching
lecture videos, another task state may be to work on quizzes, and yet
another may be to participate in forum discussions. While in each of these
different states, the student would tend to exhibit different surface
``micro'' behaviors. For example, in the lecture study state, the student
would tend to have many video-watching related behaviors and occasionally
forum activities, while in the quiz-taking state (in order to pass each
module), the student would tend to show many quiz-related ``micro''
activities as well as asking questions or checking discussions on the
forum. Note that due to the complexity of the student behavior, it is very
difficult to accurately {\em prescribe} the specific surface ``micro''
behavior patterns for each state in advance, especially without  prior knowledge about the students. For example, forum activities are likely interleaved with other activities   in every task state and the interleaving pattern can be somewhat irregular with potentially many variations. 
The major motivations for using an HMM are that 
\begin{enumerate*}[label=(\arabic*)]
    \item it uses a probabilistic model (i.e., the output probability distribution $p(x|u)$ conditioned on each state) to directly capture the inevitable uncertainty in the association of surface ``micro'' activities with their corresponding latent task/behavior state, which is often our main target to discover and characterize, and 
    \item it does not make any assumption about which latent task/behavior
      state must be associated with which observed activities or how a
      student would move from one state to another, but instead allows our
      data to ``tell'' us what kind of associations are most likely,
      what kind of transitions are most probable, and which states tend to be more long-lasting for any particular set of students. 
\end{enumerate*}

However, if we use an ordinary HMM to analyze our data, we would treat each
observed ``micro'' activity (e.g., video watching, or forum post reading)
as an output symbol, and thus the output distribution $p(x|u)$ for each
discovered latent state would be a simple distribution over all kinds of
observable micro activities recorded in our log data (e.g., 50\% lecture
watching, 8\% quiz taking, 7\% quiz submission, 2\% course wiki reading,
...). While such a distribution is meaningful and can already help us
interpret the corresponding latent state, it only gives us a rather
superficial characterization of student behavior.

Ideally, we want $p(x|u)$ to characterize the directly observable ``micro''
behavior in more detail to further capture the relations and dependencies
of these micro activities. To this end, we would treat an {\em entire
sequence} of micro activities (e.g., one session of activities) as an
observed ``symbol'' from a latent state, and further model the generation
of such a sequence with another Markov model where we treat each micro
activity as an {\em observable} state, and model the transitions between
these activity states in very much the same way as the state transitions in
HMM.

Adding this second layer would allow us to characterize a latent task
state in much more detail, as it would reveal not only what activities are
most common to a task state, but also the transition patterns between these
``micro'' activities (e.g., it can reveal frequent back-and-forth
transitions between quiz-taking and quiz-submission, which would suggest a
concentrated period of taking quizzes). Combining this ``surface'' Markov
model over the ``micro'' actions with the ``deep'' hidden Markov model over
the latent task states gives us a fairly general and powerful two-layer HMM
(TL-HMM) that can simultaneously learn ``deeply'' the latent task/behavior
states and their transitions as well as the corresponding ``micro''
activity transition patterns associated with each latent state to
facilitate interpretation and analysis of the discovered latent state
patterns. Our implementation of the learning algorithm for TL-HMMs is
included as part of the \meta/ toolkit~\citemeta{}.


\bibliography{bib}
\end{document}
